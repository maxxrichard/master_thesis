% !TEX root = mythesis.tex

%==============================================================================
\chapter{Introduction}
\label{sec:intro}
%==============================================================================
\textit{“Not only is the Universe stranger than we think, it is stranger than we can think.”} - Werner Heisenberg

\vspace{0.5cm}
The formation of the universe is explained by cosmological and particle physics models. The field of high energy physics tries to understand the nature and the formation of the universe. It was started in 1897 when the discovery of electron~\cite{electron} let humanity to step into the subatomic scale. Later on, the quantum mechanics and general relativity were established well enough to answer the open questions which pushed the boundaries of the scientific limits further. The two theories which can precisely describe the universe are the Standard Model on particle physics (SM) and the Lambda Cold Dark Matter ($\Lambda$CDM) model of cosmology. These theories are confirmed to high precision by the Large Hadron Collider (LHC) and the Planck Mission~\cite{planck} experiments. The LHC was built from 1998 to 2008~\cite{lhc} to collide high-energy protons to probe the behaviour of the particles.

A great milestone in the history of particle physics was reached when the Higgs boson was discovered by the ATLAS~\cite{higgsatlas} and CMS~\cite{higgscms} experiments at the LHC in 2012. It was the last piece of the Standard Model to be discovered. With this, the Standard model is considered to be the most fundamental theory, which is accurate and precise in all of its predictions. However, the fact is that it is still not a complete theory because of the limitation of not able to explain some important phenomena. One of them is the instability of the mass of the Higgs boson. The radiative corrections to the mass of the Higgs boson conflict with the experimental value of the Higgs boson mass, which is $\SI{125.7}{\giga\electronvolt}$~\cite{higgsatlas}~\cite{higgscms}. These corrections quadratically diverge up to the cut-off scale of new physics. This is known as the Higgs mass hierarchy problem. This leads physicists to think beyond the scope of the Standard Model. There are various theories which are proposed to explain this phenomenon. Some of these theories predict the existence of an exotic particle called vector-like quark (VLQ) which couples with the Standard Model third generation quarks. VLQ can solve the Higgs mass hierarchy problem by introducing an additional loop correction which cancels out the divergence of these corrections. The search for VLQ is carried out in both ATLAS and CMS data collected so far from $pp$ collisions at the LHC. VLQ can either be produced singly or in pair. The single VLQ production is a dominant process at high VLQ masses~\cite{wulzer}. In this thesis, the single production of two such types of VLQs, $T$ and $Y$ quark are presented in $T/Y\rightarrow Wb$ analysis.

The main topic discussed in this thesis is the background estimation in the all-hadronic channel of $T/Y\rightarrow Wb$ analysis by using a data-driven method. This thesis is organised as follows. A brief introduction of the theoretical concepts which are important in particle physics is given in Chapter \ref{sec:theory}. A description of the Standard Model is provided and familiarises the reader with the idea of vector-like quarks. In Chapter \ref{sec:lhcandatlas}, an overview of the LHC accelerator and a detailed description of the ATLAS detector are given. Jets and its reconstruction algorithm are discussed in Chapter \ref{sec:jetsandtaggers}. The two different types of jet algorithms are also discussed in detail which are used in this thesis. In Chapter \ref{sec:analysisstrategy}, the strategy used for this analysis is described, including the study of all the possible backgrounds, data preparation for the analysis and the selection of interesting events. The topics essential to physics analyses, such as a description of the data and Monte Carlo samples used and reconstruction of physics objects with the ATLAS detector are also covered. An introduction to a data-driven method called the ABCD method is given in Chapter \ref{sec:abcd}. The ABCD method is used to estimate the multijet background.  An overview of the method and its implementation on the data along with the further approaches to improve the estimate are shown in this chapter. In Chapter \ref{sec:uncertainty_result}, all the uncertainties which are taken into account while performing the multijet estimation are described, which include both statistical and systematic uncertainties. In Chapter \ref{sec:results}, the final result of the multijet estimate, including all the uncertainties is presented. A comparison between the performances of different $W$-taggers and the jet collections are also shown. Finally, in Chapter \ref{sec:conclusion}, a brief summary and the concluding remarks about the research that is documented in this thesis are provided.


%%% Local Variables: 
%%% mode: latex
%%% TeX-master: "mythesis"
%%% End: 
